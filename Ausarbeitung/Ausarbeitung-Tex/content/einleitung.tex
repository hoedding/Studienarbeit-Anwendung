\section{Einleitung}
Diese Studienarbeit wird im Zuge des Studiums Bachelor of Engineering - Informationstechnik an der DHBW Karlsruhe erstellt. \\\\
In dieser Arbeit wird ein Komplettsystem entwickelt, welches sowohl die Überwachung, als auch die Steuerung der Beleuchtung von Innen- und Außenbereichen ermöglicht. Die universelle Einsetzbarkeit und leichte Konfigurierbarkeit sind die Hauptaspekte bei der Entwicklung.\\\\
Die Beleuchtung wird mit adressierbaren LED-Pixeln umgesetzt, da diese sehr leicht steuer- und erweiterbar sind. Für die Erkennung von Bewegungen werden klassische Bewegungsmelder eingesetzt. Das Aufrufen und Speichern der Überwachungsbildern wird mit einer Kamera und einem FTP-Server realisiert. Die gesamte Steuerung funktioniert mittels einer iOS\footnote{Das System iOS ist das Betriebssystem aller Mobiltelefone der Firma Apple.}-Application. Die Informationen werden über ein Netzwerk an den Raspberry Pi übertragen. Die Implementierung dieser App\footnote{App wird im Laufe der Arbeit als Akürzung für mobile Anwendungen (Application) benutzt.} erfolgt in Swift und die der Server-Anwendung in Python.\\\\
Es müssen passende Bauteile und Produkte evaluiert und getestet werden. Diese müssen vom Raspberry Pi\footnote{Der Raspberry Pi ein Einplatinencomputer (ARM) der Raspberry Pi Foundation.} ansteuerbar sein. Des Weiteren ist ein Teil der Arbeit die Ausarbeitung der Architektur der Serveranwendung und der iOS-App. Zur Fertigstellung müssen beide Anwendungen implementiert werden und die funktionsfähige Anwendung an einem Beispielobjekt in Betrieb genommen werden.\\\\
Es gibt drei verschiedene Modi in denen sich das System befinden kann:
\begin{itemize}
\item Beleuchtung wird durch Bewegungsmelder ausgelöst
\item Beleuchtung wird manuell vom Benutzer über App gesteuert 
\item Bewegungsmelder als Alarmanlage, beim Auslösen wird der Benutzer benachrichtigt und ein Bild der Kamera als Notification auf dem Smartphone angezeigt
\end{itemize}
Der Autor dieser Arbeit hatte zu Beginn ihrer Erstellung keine Erfahrung im Bezug auf Programmierung mit Python / Swift und der Kommunikation über Netzwerke mit diesen Sprachen. Somit müssen die Grundlagen und alle erforderlichen Fähigkeiten in diesen Sprachen angeeignet werden. \\\\

\section{Hinweise}
\paragraph{Literatur}
Um die Programmiersprachen an sich zu erlernen, wurde die folgende Literatur genutzt. Die neu erlernten, in diesem Projekt benutzten Sprachfeatures werden nicht alle mit einzelnen Quellen versehen. 
\begin{itemize}
	\item \textbf{Swift} Zur Einarbeitung in Apples' Programmiersprache Swift wurde das Buch 'Swift im Detail' von Thomas Sillmann aus dem Verlag Hanser genutzt.\\
	ISBN: 978-3-446-44294-8\\
	Auflage: 2015
	
	\item \textbf{Python} Zur Einarbeitung in Python wurde das Buch 'Introducing Python' von Bill Lubanovic aus dem Verlag O'Reilly genutzt. \\
	ISBN: 978-1-449-35936-2\\
	Auflage: 2015
	
	\item \textbf{Raspberry Pi} Um den Umgang mit Programmierung auf dem Raspberry Pi zu erlernen, wurde das Buch 'Raspberry Pi programmieren mit Python' von Michael Weigend aus dem Verlag mitp genutzt. \\
	ISBN: 978-3-8266-9474-5\\
	Auflage: 2014
\end{itemize}

\paragraph{Grafiken} Sofern nicht anders gekennzeichnet, wurden alle Grafiken von Timo Höting erstellt.

\paragraph{Quellcode} Der implementierte Code und die ausformulierte Studienarbeit sind sowohl auf der beigelegten CD, als auch auf Github unter \url{https://github.com/hoedding/Studienarbeit-Anwendung} verfügbar.

\section{Projektmanagement}
\subsection{Meilensteintrendanalyse}
Die Meilensteintrendanalyse ist eine Art des Projektmanagements. Hauptaufgabe ist die Überwachung des Projektfortschritts und die frühe Erkennung von Terminverzögerungen. Hierfür werden bei Projektbeginn Meilensteine festgelegt, die Inhalt, Dauer und Endzeitpunkt enthalten. Im Laufe eines Bearbeitungszeitraums können mögliche Verzögerungen erkannt und entsprechend darauf reagiert werden. Um große Verzögerungen zu vermeiden, sollten realistische Sicherheitspuffer eingeplant werden. \\
Bei Beendigung eines Meilensteins kann ein Fazit aus dessen Ablauf gezogen werden. Zum Beispiel kann bei aufgetretener Verzögerung Ursachenforschung betrieben werden, um in weiteren Schritten solche Verzögerungen zu vermeiden. 

\paragraph{Bemerkung}
\begin{itemize}
\item Die schriftliche Ausarbeitung ist nicht Teil der Meilensteine. Sie erfolgt parallel zu den durchgeführten Aufgaben.
\end{itemize}

\subsection{Meilensteine und Erfolgsprüfung}
\begin{enumerate}

\item Planung der Architektur (15. September - 30. September 2014)
\begin{itemize}
\item Entwurf Anwendungsstruktur
\begin{itemize}
\item Der Entwurf der Anwendungssturktur für Serverimplementierung und Mobile-Implementierung konnte erfolgreich abgeschlossen werden.
\end{itemize}
\item Ermittlung notwendiger Hardware für die einzelnen Anwendungsfälle
\begin{itemize}
\item Dies benötigte nur geringen Aufwand, da nur wenige Bauteile benötigt werden. Die Evaluierung, Beschaffung und Tests fällt in den 3. Meilenstein.
\end{itemize}
\item Ausarbeitung Übertragungsprotokoll
\begin{itemize}
\item Das Protokoll wurde erfolgreich ausgearbeitet. Die Details sind in 2.5.1 dargestellt.
\end{itemize}
\item Einarbeitung in Python
\begin{itemize}
\item Es wurden die Grundlagen der Sprache Python im Bezug auf OOP, Funktionen und Datentypen erarbeitet. Die Kenntnisse haben sich im Laufe des Projekts weiter verbessert. 
\end{itemize}
\end{itemize}

\item Funktionsfähiger Prototyp Webserver (1. Oktober - 19. Oktober 2014)
\begin{itemize}
\item Auswahl eines Frameworks für die Implementierung des Webservers
\begin{itemize}
\item Es wurde das Twisted Framework ausgewählt und Testimplementierung der verschiedenen Server-Typen (Socket, SSL, STARTTLS, HTTP, HTTPS) durchgeführt. 
\end{itemize}
\item Implementierung der für den Webserver nötigen Klassen
\begin{itemize}
\item Implementierung der Socketübertragung. Im Laufe des Projekts zeigte sich, dass dies nicht die optimale Lösung ist. in einem späteren Meilenstein wurde ein HTTPS-Webserver mit Twisted implementiert. 
\end{itemize}
\item Testen der Funktionen
\begin{itemize}
\item Testen mithilfe von Clients, die in Python implementiert wurden. 
\end{itemize}
\end{itemize}

\item Auswahl Hardwarekomponenten (LEDs, Sensoren, Kamera) (20. Oktober - 31. Oktober 2014)
\begin{itemize}
\item Evaluierung
\begin{itemize}
\item Die Evaluierung von LEDs und Sensoren konnte erfolgreich abgeschlossen werden. Für die Wahl der Netzwerkkamera konnte keine Lösung gefunden werden, da noch nicht klar war, in welcher Form die Bilder abgerufen werden können. Dieser Aufgabenteil ist in Meilenstein 7 verschoben worden.
\end{itemize} 
\item Beschaffung
\begin{itemize}
\item Die Beschaffung von LEDs und Sensoren verlief mit erfolgreich mit geringem Aufwand. 
\end{itemize}
\item Testen und Testimplementierung
\begin{itemize}
\item Die Testimplementierungen erfolgreich durchgeführt werden. Der Quellcode und die zugehörigen Schaltbilder befinden sich in den Punkten 2.1 und 2.2.
\end{itemize}
\end{itemize}

\item Implementierung Serveranwendung (1. November - 30. November 2014)
\begin{itemize}
\item Implementierung Sensorerkennung
\begin{itemize}
\item Die Implementierung der Sensorerkennung war erfolgreich.
\end{itemize}
\item Implementierung Ansteuerung LED
\begin{itemize}
\item Die Implementierung der Ansteuerung der LEDs war erfolgreich.
\end{itemize}
\item Implementierung der Konfigurationsmöglichkeiten ( hat nicht geklappt -> Dezember, Januar)
\begin{itemize}
\item Die vollständige Implementierung der Konfigurationsdateien, -lesern und -schreibern, sowie der Übertragung wurde nicht abgeschlossen. Grund dafür war fehlende Kenntniss über den Empfang auf dem Client und über JSON.
\end{itemize}
\item Implementierung des Übertragungsprotokolls
\begin{itemize}
\item Die Auswertung der empfangenen Daten wurde erfolgreich implementiert. 
\end{itemize}
\item Komplette Implementierung 
\begin{itemize}
\item Die Implementierung der gesamten Serveranwendung wurde soweit fertiggestellt, dass ein Betrieb möglich war. Einige kleine Änderungen oder nachträgliche Erweiterungen wurden im Laufe des Projekts hinzugefügt. Dies waren meistens Dinge, die vorher nicht bedacht wurden, oder an die mobile App angepasst werden mussten.
\end{itemize}
\end{itemize}

\item Funktionsfähiger Prototyp iOS-Anwendung (1. Januar - 31. Januar 2015)
\begin{itemize}
\item Einarbeitung Swift und XCode
\begin{itemize}
\item In dieser Zeitphase wurde der Umgang mit XCode und der Sprache Swift erlernt. 
\end{itemize}
\item Erstellung Prototyp der Anwendung in XCode
\begin{itemize}
\item Es wurde die Anwendungsstrktur in XCode erstellt. 
\end{itemize}
\item Auswahl von nötigen Frameworks
\begin{itemize}
\item Es wurden Frameworks für Menüführung, Sicherheit und FTP-Verbindung ausgewählt und hinzugefügt. 
\end{itemize}
\item Übertragungen mit dem Webserver
\begin{itemize}
\item In dieser Phase wurde festgestellt, dass die Übertragung über Sockets nicht optimal ist. Eine Übertragung mittels HTTP Protokoll bietet eine deutlich einfachere und sicherere Implementierung. Aufgrund dieser Feststellung wurde die Implementierung des Webservers in diesem Meilenstein verändert. 
\end{itemize}
\item Refactoring Webserver
\begin{itemize}
\item Der Webserver wurde auf HTTPS umgestellt. 
\item Viele kleine Veränderungen im Server.
\end{itemize}
\end{itemize}

\item Implementierung iOS-Anwendung (1. Februar - 31. März 2015)
\begin{itemize}
\item Server-Client Kommunikation
\begin{itemize}
\item Die Übertragung wurde entsprechend dem Anwendungsprotokoll implementiert. 
\end{itemize}
\item User-Interface
\begin{itemize}
\item Das User-Interface wurde erstellt und mit Funktionen versehen. 
\end{itemize}
\item Implementierung konsistene Speicherung
\begin{itemize}
\item Der Zugriff auf CoreData wude implementiert.
\end{itemize}
\end{itemize}

\item Abschluss der Arbeit (1. April - 11. Mai 2015)
\begin{itemize}
\item Implementierung Netzwerkkamera
\begin{itemize}
\item Die Anbindung der Netzwerkkamera war einer der aufwändigsten Punkte in diesem Projekt. Nach verschiedenen Ansätzen wurde eine gute Lösung erarbeitet.
\end{itemize}
\item Beispielobjekt
\begin{itemize}
	\item Das gesamte Projekt wurde in einem Treppenhaus installiert.
\end{itemize}
\item Fertigstellung Ausarbeitung
\item  Abgabe
\end{itemize}

\end{enumerate}

