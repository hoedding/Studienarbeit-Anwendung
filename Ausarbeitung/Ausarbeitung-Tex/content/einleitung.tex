\section{Vorwort}
\section{Projektbeschreibung}
Im Zuge dieser Studienarbeit soll ein Komplettsystem entwickelt werden, dass sowohl die Überwachung als auch die Steuerung der Beleuchtung von Innen- und Außenbereichen ermöglicht. Das System soll nach Entwicklung universell einsetzbar und leicht konfigurierbar sein.\\
Die Beleuchtung soll mit adressierbaren LED-Pixeln umgesetzt werden, da diese sehr leicht steuer- und erweiterbar sind. Für die Erkennung von Bewegungen sollen klassische Bewegungsmelder eingesetzt werden. Mittels einer Kamera sollen Bilder aufgerufen und gespeichert werden können. Die gesamte Steuerung soll mittels einer iPhone-App über einen Raspberry Pi erfolgen. Die Implementierung dieser App soll in Swift erfolgen und die der Server-Anwendung in Python.  \\
Es müssen passende Bauteile und Produkte evaluiert und getestet werden. Diese müssen vom Raspberry Pi ansteuerbar sein. Ob die Überwachungskamera direkt am Raspberry Pi angeschlossen wird oder sich nur im selben Netzwerk befindet, wird im Laufe dieses Projekts erarbeitet. 
Es gibt drei verschiedene Modi in denen sich das System befinden kann:
\begin{itemize}
\item Beleuchtung wird durch Bewegungsmelder ausgelöst (Reaktion darauf kann vom User definiert werden)
\item Beleuchtung wird manuell vom Benutzer über App gesteuert (Color Chooser, Bereichsauswahl, Leuchteffekte)
\item Bewegungsmelder als Alarmanlage, beim Auslösen wird der Benutzer benachrichtigt und Bild der Kamera als Notifcation auf dem Smartphone angezeigt
\end{itemize}

\section{Teilprojekte}
\begin{itemize}
\item LED-Pixel und Bewegungssensoren evaluieren / ansteuern
\item Implementierung der Ansteuerung / des Protokolls (mit den drei verschiedenen Modi)
\item Implementierung der iOS App
\item Vollständige praktische Umsetzung an einem Beispielobjekt
\end{itemize}
