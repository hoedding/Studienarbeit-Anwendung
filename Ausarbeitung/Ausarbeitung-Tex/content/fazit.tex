Die praktische Umsetzung in der letzten Phase des Projekts hat gezeigt, dass ein funktionierendes und leicht einzusetzendes System entwickelt wurde. Alle nötigen Bauteile wurden evaluiert und in Teststellungen auf ihre Funktionalität überprüft. Die gewünschte Überwachung ist durch die Bewegungssensoren gegeben und wird zuverlässig von der Serveranwendung ausgewertet. Des weiteren können sämtliche Funktionen der Serverfunktion mit einer iOS-Application gesteuert und konfiguriert werden. Somit ist die Funktionsfähigkeit der Anwendung auch ohne physikalischen Zugriff auf den Raspberry Pi gegeben. Nach einigen Anläufen (und kurzen Nächten) wurde auch die Auswertung der Kamerabilder gelöst und anforderungsgemäß in den Anwendungen implementiert.  \\\\
Das geplante Projektmanagement mittels Meilensteinen konnte kontinuierlich umgesetzt werden. Aufgaben, bei denen Schwierigkeiten aufgetreten sind konnten ohne große Verzö- gerung in spätere Meilensteine verschoben werden. Durch den Einsatz dieses Projektmanagements war zu jedem Zeitpunkt ein Überblick über die noch zu erledigenden Aufgaben möglich. \\\\
Als größtes Ziel in dieser Projektarbeit ist das Erlernen der beiden Sprachen hervorzuheben. Es stellt eine große Herausforderung dar, ohne Vorkenntnisse eine Anwendung zu entwickeln, die auf verschiedenen Plattformen ausgeführt wird und über ein Netzwerk kommuniziert. Auch konnten viel Erfahrungswerte im Umgang mit Linuxbetriebssystemen und der Entwicklungsumgebung von Apple gesammelt werden.\\\\
Laut den Richtlinien der Dualen Hochschule Baden-Württemberg für Studienarbeiten soll '[...] neben der fachlichen Auseinandersetzung mit dem
gestellten Thema insbesondere das eigenverantwortliche Einarbeiten in eine neue Themenstellung [...] im Vordergrund stehen.' \cite{dhbw} Diese Anforderung ist somit erfüllt.\\\\
Somit wurden alle aufgestellten Ziele im gegebenen Zeitraum erledigt.\\
\begin{quote}
	\centering
	Lernen ist Erfahrung. Alles andere ist einfach nur Information. \\
	Albert Einstein
\end{quote}