\subsection{Installation}
Eine gesonderte Installation ist nicht notwendig, es ist nur erforderlich die Anwendung aus Git zu klonen. 
\begin{lstlisting}[caption =Git Clone der Studienarbeit, language=python, frame=single, breaklines=true,columns=fullflexible, commentstyle=\color{gray}\upshape, captionpos=b]
git clone https://github.com/hoedding/Studienarbeit-Anwendung.git
\end{lstlisting}

\subsection{Konfiguration}
Die Konfiguration der Anwendung erfordert nur das Ausfüllen der Konfigurationsdatei. Dies ist leicht mit einem Frage-Antwort-Dialog umzusetzen. Dieser wird ebenfalls in Python implementiert. Für das Schreiben der Konfiguration kann die schon vorhandene ConfigWriter-Klasse verwendet werden. 
\begin{lstlisting}[caption =Implementierung setup.py, language=python, frame=single, breaklines=true,columns=fullflexible, commentstyle=\color{gray}\upshape, captionpos=b, numbers = left]
def raspberryConfig(self):
    writer = ConfigWriter()
    print '##########    Konfiguration Raspberry Pi  ###########'
    print '#######  Benutzer: admin                     ########'
    print '#######  Passwort: password                  ########'
    ledport = raw_input("Port der LED? ")
    ledcount = raw_input("Anzahl der angeschlossenen LEDs? ")
    motionport1 = raw_input("Port Bewegungssensor 1? ")
    motionport2 = raw_input("Port Bewegungssensor 2? ")
    timer  = raw_input("Zeitdauer bei Bewegungsmelder? ")
    writer.changeConfig("username", "armin")
    writer.changeConfig("pw", "d63dc919e201d7bc4c825630d2cf25fdc93d4b2f0d46706d29038d01")
    writer.changeConfig("ledport", ledport)
    writer.changeConfig("ledcount", ledcount)
    writer.changeConfig("motionport1", motionport1)
    writer.changeConfig("motionport2", motionport2)
    writer.changeConfig("timeperiod", timer)
\end{lstlisting}
Im Anschluss müssen noch die erforderlichen Zertifikate (Server-Zertifikat und APN-Zertifikat) erzeugt werden und in den Ordner 'certs' gespeichert werden. 
