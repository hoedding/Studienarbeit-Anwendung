\documentclass[12pt,a4paper]{article}
\usepackage[ngerman]{babel} 
\usepackage[utf8]{inputenc} 
\usepackage{fancyhdr}

\usepackage{pdfpages}
\usepackage{url}
\usepackage{graphicx}
\usepackage{wrapfig}
\usepackage{geometry} 
\geometry{a4paper, top=25mm, left=25mm, right=25mm, bottom=25mm,  footskip=12mm}
\usepackage{textcomp}

\pagestyle{fancy}
\fancyhf{}

\usepackage{titlesec}
\newcommand{\sectionbreak}{\clearpage}

% Header für Seiten ohne Chapter
\fancyhf{}
\fancyhead[L]{Studienarbeit Timo Höting \\ DHBW Karlsruhe}
\fancyhead[R]{\includegraphics[scale=0.3]{./data/dhbwlogo.jpg} } 
\fancyfoot[C]{\thepage}

\fancypagestyle{plain}{
	% Header für Seiten mit Chapter
	\fancyhf{}
	\fancyhead[L]{Studienarbeit Timo Höting \\ DHBW Karlsruhe}
	\fancyhead[R]{\includegraphics[scale=0.3]{./data/dhbwlogo.jpg} } 
	\fancyfoot[C]{\thepage}
}

\usepackage{listings}
\lstset{literate=%
	{Ö}{{\"O}}1
	{Ä}{{\"A}}1
	{Ü}{{\"U}}1
	{ü}{{\"u}}1
	{ä}{{\"a}}1
	{ö}{{\"o}}1
	{~}{{\textasciitilde}}1
}

\begin{document}
\begin{titlepage}
\begin{figure}
\makebox[\textwidth]{\includegraphics[page={1},width=\paperwidth]{./data/deckblatt.pdf}} \\
\end{figure}
\end{titlepage}
\clearpage
\begin{figure}
\makebox[\textwidth]{\includegraphics[page={2},width=\paperwidth]{./data/deckblatt.pdf}} \\
\end{figure}
\thispagestyle{empty} %Head löschen
\tableofcontents
\thispagestyle{empty}  
\chapter{Einleitung} \section{Vorwort}
\section{Projektbeschreibung}
Im Zuge dieser Studienarbeit soll ein Komplettsystem entwickelt werden, dass sowohl die Überwachung als auch die Steuerung der Beleuchtung von Innen- und Außenbereichen ermöglicht. Das System soll nach Entwicklung universell einsetzbar und leicht konfigurierbar sein.\\
Die Beleuchtung soll mit adressierbaren LED-Pixeln umgesetzt werden, da diese sehr leicht steuer- und erweiterbar sind. Für die Erkennung von Bewegungen sollen klassische Bewegungsmelder eingesetzt werden. Mittels einer Kamera sollen Bilder aufgerufen und gespeichert werden können. Die gesamte Steuerung soll mittels einer iPhone-App über einen Raspberry Pi erfolgen. Die Implementierung dieser App soll in Swift erfolgen und die der Server-Anwendung in Python.  \\
Es müssen passende Bauteile und Produkte evaluiert und getestet werden. Diese müssen vom Raspberry Pi ansteuerbar sein. Ob die Überwachungskamera direkt am Raspberry Pi angeschlossen wird oder sich nur im selben Netzwerk befindet, wird im Laufe dieses Projekts erarbeitet. 
Es gibt drei verschiedene Modi in denen sich das System befinden kann:
\begin{itemize}
\item Beleuchtung wird durch Bewegungsmelder ausgelöst (Reaktion darauf kann vom User definiert werden)
\item Beleuchtung wird manuell vom Benutzer über App gesteuert (Color Chooser, Bereichsauswahl, Leuchteffekte)
\item Bewegungsmelder als Alarmanlage, beim Auslösen wird der Benutzer benachrichtigt und Bild der Kamera als Notifcation auf dem Smartphone angezeigt
\end{itemize}

\section{Teilprojekte}
\begin{itemize}
\item LED-Pixel und Bewegungssensoren evaluieren / ansteuern
\item Implementierung der Ansteuerung / des Protokolls (mit den drei verschiedenen Modi)
\item Implementierung der iOS App
\item Vollständige praktische Umsetzung an einem Beispielobjekt
\end{itemize}

\chapter{Hauptteil}
\section{LED-Pixel} 
\subsection{Bewertungskriterien}
Die Beleuchtung soll durch einzelne LED-Pixel stattfinden. Ein Pixel bedeutet ein Chip auf dem sowohl die LED und der nötige Treiber sitzt. Für die Evaluierung werden folgende Kriterien gewählt:
\begin{itemize}
\item RGB-Farbraum \\
Die LED muss den gesamten RGB-Farbraum darstellen können. \\
Gewichtung: 5, KO-Kriterium
\item Ansteuerung \\
Da der Raspberry Pi an einigen seiner Pins Pulsweitenmodulation (PWM) bietet, sollten die LED-Pixel ohne extra Hardware ansteuerbar sein. Eine extra Stromversorgung ist aber bei größerer Anzahl an LEDs unabdingbar. \\
Gewichtung: 10
\item Framework \\
Hier wird bewertet ob der jeweilige Hersteller ein fertiges Framework zu seinen Produkten anbietet. \\
Gewichtung: 10
\item Kosten \\
Es werden nur die reinen Produktkosten, also ohne Versand und Zoll, bewertet. \\
Gewichtung: 5
\item Extras \\
An dieser Stelle können mögliche Extras eines Herstellers einfließen. \\
Gewichtung: 5
\end{itemize}

\subsection{Evaluierung}
\begin{itemize}
\item Adafruit, Neopixel \\
https://www.adafruit.com/neopixel \\
LED-Pixel in unzähligen Ausführungen. \\
Sitz der Firma in Tampa, Florida, USA \\
RGB: Chip ist der WS2801, http://www.adafruit.com/datasheets/WS2801.pdf -> Hat volle Abdeckung des RGB-Farbraums \\
Ansteuerung: Findet über PWM-Pin des Raspberry Pi statt. \\
Framework: Framework von Adafruit, welches eine sehr leichte Ansteuerung ermöglichen soll. \\
Kosten: 4 LEDs  7\$, 25 LEDs zusammen  39\$, durch Lieferung aus USA sehr hohe Versandkosten (50\$) \\
Extras: Händler bietet verschiedene Formen und fertige Ketten an. \\
\item LED-Emotion GMBH, LED Streifen \\
http://www.led-emotion.de/de/LED-Streifen-Set.html \\
LED-Streifen, keine Einzelpixel, nur mit Controller, keine API \\
RGB: Voller RGB-Farbraum \\
Ansteuerung: Nur mit Controller  \\
Framework: Keine öffentliche Api, möglicherweise mit Raspberry Pi ansteuerbar  \\
Kosten: 30 LEDs mit Netzteil 79€   \\
Extras: keine
\item DMX4ALL GmbH, MagiarLED Solutions \\
http://www.dmx4all.de/magiar.html \\
Spezialisiert auf DMX-Ansteuerung, keine öffentliche API \\
RGB: Volle Abdeckung RGB-Farbraum \\
Ansteuerung: Wird über DMX-Controller angesteuert, dieser setzt die Signale um. \\
Framework: DMX-Ansteurung über DMX-Controller \\
Kosten: Streifen mit 72 LEDs = 99€ \\
Extras: viele verschiedene Varianten
\item TinkerForge, RGB LED-Pixel \\
https://www.tinkerforge.com/de/shop/accessories/leds.html \\
Scheinen die gleichen wie von Adafruit zu sein, allerdings werden hauptsächlich Controller im Shop angeboten \\
RGB: Chip WS2801, volle Abdeckung RGB-Farbraum \\
Ansteuerung: Nach Anfrage an den Anbieter sollen die LEDs baugleich zu denen von Adafruit sein.  \\
Framework: keins, aber Ansteuerung über das Framework von Adafruit \\
Kosten: 50 LEDs = 59€ \\
Extras: Lieferung aus Deutschland
\end{itemize}
\begin{minipage}{\linewidth}
            \centering
            \includegraphics[width=\textwidth]{./data/evaluierung-led.png}
            \captionof{figure}{Ergebnisse der LED-Evaluierung}
        \end{minipage}
\paragraph{Fazit:}
In der Evaluierung schneiden die Produkte von Adafruit und TinkerForge am besten ab. Für eine erste Teststellung werden die einzelnen LED-Pixel von Adafruit aus den USA bestellt (Neopixel). An diesen soll vor allem die Ansteuerung getestet werden. Falls sie sich bewähren, wird für den endgültigen Aufbau auf die LED-Ketten von Tinkerforge zurück gegriffen. 

\subsection{Teststellung}
Für einen ersten Test wurde das in XXX ausgewählte Produkt als einzelne Pixel bestellt. Der Hersteller Adafruit bietet hier 4er-Packungen an. Diese können leicht in eigene Schaltungen eingelötet oder auf Experimentier-Boards gesteckt werden. Bei geringer Anzahl LEDs reicht die 5V-Stromversorgung des Raspberry Pi aus. 
\paragraph{Technische Daten Neopixel:} 
	\begin{itemize}
	\item Maße: 10.2mm x 12.7mm x 2.5mm
	\item Protokollgeschwindigkeit: 800 kHz
	\item Spannung: 5-9VDC  (bei 3,5V gedimmte Helligkeit) 
	\item Strom: 18,5mA / LED, 55mA / Pixel
	\end{itemize}
\paragraph{Framework:}
	\begin{itemize}
	\item RPI\_WS281X (https://github.com/jgarff/rpi\_ws281x)
	\item Sprache: Python
	\item Entwickelt für Raspberry Pi
	\item Vorraussetzung: Python 2.7
	\end{itemize}
\paragraph{Ablauf des Tests:}
\begin{itemize}

\item \textbf{Aufbau der Schaltung}\\
An die einzelnen LED-Pixel wurden Stecker angelötet, damit sie auf das Experimentierboard aufgesteckt werden können. Dann wird die Schaltung nach folgendem Schaltbild verbunden. Wichtig ist, dass beim Raspberry Pi nur Pins verwendet werden können, welche PWM\footnote{Pulsweitenmodulation: Signalübertragung durch Wechsel zwischen zwei Spannungen (High, Low), Breite des Impulses ist das Signal} bieten. \\
\begin{minipage}{\linewidth}
            \centering
            \includegraphics[width=8cm]{./data/TestSchaltungLED.png}
            \captionof{figure}{Schaltung für LED-Test}
        \end{minipage}
\item \textbf{Installation des Frameworks} 
\begin{lstlisting}[caption = Installation Framework ws281x, language=Python, frame=single, breaklines=true,columns=fullflexible, commentstyle=\color{gray}\upshape, captionpos=b, numbers = left]
wget https://github.com/tdicola/rpi_ws281x/raw/master/python/dist/rpi_ws281x-1.0.0-py2.7-linux-armv6l.egg 
sudo easy_install rpi_ws281x-1.0.0-py2.7-linux-armv6l.egg
\end{lstlisting}
\item \textbf{Testcode}

\begin{lstlisting}[caption = Testcode zur Ansteuerung der LEDs, language=python, frame=single, breaklines=true,columns=fullflexible, commentstyle=\color{gray}\upshape, captionpos=b, numbers = left]
from neopixel import * 
	
LED_COUNT   = 4       # Number of LED pixels. 
LED_PIN     = 18      # GPIO pin connected to the pixels (must support PWM!).
LED_FREQ_HZ = 800000  # LED signal frequency in hertz (usually 800khz)
LED_DMA     = 5       # DMA channel to use for generating signal (try 5)
LED_INVERT  = False   # True to invert the signal (when using NPN)

strip = Adafruit_NeoPixel(LED_COUNT, LED_PIN, LED_FREQ_HZ, LED_DMA, LED_INVERT)

strip.begin()
strip.setPixelColor(0, Color(255, 255, 255))
strip.setPixelColor(1, Color(255, 255, 255))
strip.setPixelColor(2, Color(255, 255, 255))
strip.setPixelColor(3, Color(255, 255, 255))
strip.show()
\end{lstlisting}
\end{itemize}
\paragraph{Fazit}
Die einzelnen Pixel sind sehr leicht anzusteuern, unterstützen auch das automatische Abschalten nach einer bestimmten Zeit und haben eine sehr hohe Leuchtkraft. Die Evaluierung hat zu einer guten Produktwahl geführt. \\ 
Nach einer weiteren Nachfrage an Tinkerforge wurde versichert, dass deren LED-Ketten Baugleich zu denen von Adafruit sind. Aufgrund der hohen Versandkosten werden für die endgültige Teststellung die Produkte von Tinkerforge gewählt.

\section{Bewegungssensor} In einem der Modi soll die Beleuchtung durch den Bewegunsmelder ausgelöst werden. Hierfür sind zuverlässige und weitreichende Bewegungssensoren notwendig.
\subsection{Bewertungskriterien}
\begin{itemize}
\item \textbf{Ansteuerung}\\
Die Anbindung an den Raspberry Pi soll möglichst leicht realisierbar sein. Wünschenswert ist, dass der Sensor einfach ein High-Signal bei Bewegungserkennung ausgibt. \\
Gewichtung: 5, KO-Kriterium
\item \textbf{Reichweite}\\
Die Reichweite oder Sensivität des Sensors soll ausreichend und regelbar sein.\\
Gewichtung: 3
\item \textbf{Kosten}\\
Es werden nur die reinen Produktkosten, also ohne Versand und Zoll, bewertet. \\
Gewichtung: 1
\item \textbf{Extras}\\
An dieser Stelle können mögliche Extras eines Herstellers einfließen.\\
Gewichtung: 3
\end{itemize}

\subsection{Evaluierung}
\begin{itemize}
\item \textbf{PIR (MOTION) Sensor, Adafruit}\\
Link: http://www.adafruit.com/product/189\\
Ansteuerung: Gibt High-Signal an einem Pin aus.\\
Reichweite: 7m, 120 Grad\\
Kosten: 9,95\$ + Versand aus USA\\
Extras: Kabel inklusive\\
\item \textbf{PIR Infrared Motion Sensor (HC-SR501)}\\
Link: https://www.modmypi.com/pir-motion-sensor\\
Ansteuerung: Gibt High-Signal an einem Pin aus.\\
Reichweite: 5-7m, 100 Grad\\
Kosten: 2,99\$ + Versand aus UK\\
Extras: keine\\
\item \textbf{Infrarot PIR Bewegung Sensor Detektor Modul}\\
Link: http://www.amazon.de/Pyroelectrische-Infrarot-Bewegung-Sensor-Detektor/dp/B008AESDSY/ref=pd\_cp\_ce\_0\\
Ansteuerung: Gibt High-Signal an einem Pin aus.\\
Reichweite: 7m, 100 Grad\\
Kosten: 5 Stück = 7,66€\\
Extras: keine\\
\end{itemize}
\begin{minipage}{\linewidth}
            \centering
            \includegraphics[width=\textwidth]{./data/evaluierung-ms.png}
            \captionof{figure}{Ergebnisse der Motion-Sensor-Evaluierung}
        \end{minipage}
\paragraph{Fazit}
Die meisten Infarot-Bewegungssensoren sind von der Bauweise nahezu identisch. Die Unterschiede liegen meist nur in der Empfindlichkeit. Da die Reichweite in diesem Fall nicht von großer Bedeutsamkeit ist, kann eigentlich jedes der Produkte bestellt werden. Auf Ebay und Amazon ist die Anzahl angebotener Sensoren nahezu unbegrenzt, es wurde für die Teststellung also die oben evaluierte Variante von Amazon bestellt. 
\subsection{Teststellung}
Der in Punkt X.X.X gewählte Bewegungssensor wurde beim Hersteller bestellt. In der Teststellung reicht die Stromversorgung des Raspberry Pi. 
\paragraph{Technische Daten Sensor:}
\begin{itemize}
\item Die Empfindlichkeit und Haltezeit kann eingestellt werden
\item Reichweite: ca. 7m
\item Winkel: 100 Grad
\item Spannung: DC 4,5V- 20V
\item Strom: < 50uA
\item Ausgansspannung: High 3V / Low 0V
\item Größe: ca. 32mm x 24mm
\end{itemize}

\paragraph{Ablauf des Tests:}
\begin{itemize}
\item \textbf{Aufbau der Schaltung} \\
Der Sensor wird in der Teststellung direkt vom Raspberry Pi mit Strom versorgt. Für die Datenleitung kann jeder beliebige Pin gewählt werden. 
\item \textbf{Testcode} \\
Um eine Änderung am Datenpin festzustellen werden zwei Variable angelegt: current\_status und previous\_status. Das Programm wird in einer Dauerschleife geschickt, in der bei jedem Durchlauf die beiden Status überprüft. Wenn der neue Status (current\_status) High ist und das vorherige Signal (previous\_state) Low, dann wird eine Bewegung erkannt. Der Code wird mittels Kommentare erklärt.	

\begin{lstlisting}[caption = Testcode zur Bewegungserkennung mit Sensor, language=python, frame=single, breaklines=true,columns=fullflexible, commentstyle=\color{gray}\upshape, captionpos=b, numbers = left]
import RPi.GPIO as GPIO
import time

GPIO.setmode(GPIO.BCM)
	
# Pin definieren
MOTION_PIN1 = 7
	
# Diese als Input definieren
GPIO.setup(MOTION_PIN1,GPIO.IN)

# Status definieren um verschiedene Änderungen zu erkennen
Current_State  = 0
Previous_State = 0
					
try:
	# Loop zur Erkennung einer Bewegung
	# Sensor erkennt Bewegung -> Signal = High
	# Wartet 3 Sekunden und setzt Signal = Low
	while True :
		Current_State = GPIO.input(MOTION_PIN1)
		if Current_State == 1 and Previous_State == 0:
			print "Motion detected!"
			Previous_State=1
		elif Current_State == 0 and Previous_State == 1:
			print "Ready"
			Previous_State=0
		time.sleep(0.01)
	
except KeyboardInterrupt:
	print "Quit"
	GPIO.cleanup()
\end{lstlisting}
Bei der Endversion des Systems sollen mehrere Beweungssensoren integriert werden. Bei Auslösen des ersten Sensors sollen die LEDs angeschaltet werden und nach auslösen eines weiteren Sensors wieder ausgeschaltet werden. 
\paragraph{Auswertung}\\
Das High-Signal des Sensors lässt sich mit dem Raspberry Pi sehr leicht auswerten. Auch die Auswertung von mehreren Sensoren stellt kein Problem da. Das Ergebnis der Evaluierung konnte in dieser Tststellung bestätigt werden. 
\end{itemize}


\section{Python-Server und Protokoll} 


\subsection{Protokoll}
Um die LEDs später von einer App aus ansprechen zu können, soll ein auf Strings basierendes Protokoll implementiert werden. Hierfür muss als erstes festgelegt werden, welche Informationen übertragen werden sollen: 
\begin{itemize}
\item Authentifizierung\\
Übertragung eines Benutzers und eines Passworts. Das Passwort ist als Hashwert im System gespeichert und kann so überprüft werden. Zum Hashen wird der SHA-224-Algorithmus eingesetzt.
\item Control\\
Unterscheidung zwischen:\\
- X00: Alle LEDs ausschalten\\
- X01: Eine LED anschalten\\
- X02: LED-Bereich anschalten\\
- X03: Alle LEDs in einer Farbe anschalten\\
- X04: Effekte\\
- X05: Modus des Systems verändern\\ 
- X06: Anforderung des Systemstatus\\
- X07: Anforderung des LED-Status\\
- X08: Konfiguration ändern\\
- X09: Login überprüfen\\
         
Abhängig von diesem Feld werden die nachfolgenden Werte behandelt.
\item LED-Nummer\\
Falls nur eine LED angesprochen werden soll (Control = X00), so wird hier die Nummer angegeben. Ob sie im gültigen Range liegt wird intern überprüft.
\item Bereich Start\\
Wenn mehrere LEDs gesteuert werden sollen (Control = X01), so wird hier der Beginn des Bereichs angegeben.
\item Bereich Ende\\
Und hier das Ende des Bereichs.
\item Rot\\
Farbwert Rot 0-255
\item Grün\\
Farbwert Grün 0-255
\item Blau\\
Farbwert Blau 0-255
\item Modus\\
An dieser Stelle werden die verschiedenen Modi des Systems dargestellt.
\item Effektcode\\
Hinterlegte, fest programmierte Effekte, zum Beispiel alle LEDs anschalten in weis mit höchster Leuchstärke.
\item Konfiguration\\
Damit können einzelne Elemente der Serverkonfiguration verändert werden. Zum Beispiel die Leuchtdauer der LEDs, wenn sie durch den Bewegunsmelder ausgelöst wurden. 
\item Hash\\
Überprüfung ob die Übertragung erfolgreich war, mittels eines Hashwertes. Es wird der SHA-224-Algorithmus eingesetzt.
\end{itemize}
\textbf{Übertragungsbeispiel:}\\
\begin{lstlisting}[caption = Beispielübertragung des Protokolls, frame=single, breaklines=true,columns=fullflexible, commentstyle=\color{gray}\upshape, captionpos=b]
auth:pw:control:ledNo:rangeStart:rangeEnd:red:green:blue:modus:effectcode:config:hash
user:password:X01:0:0:49:255:255:255:::Hashvalue
\end{lstlisting}
Dies würde die LEDs 0 bis 49 einschalten (Farbe weis 255,255,255). Anstelle des 'Hashvalue' würde der Hashwert der gesamten Übertragung gesendet.

\subsection{Server-Framework}
Twisted: https://twistedmatrix.com \\
Es wird das Twisted Matrix Framework eingesetzt. Twisted ist eine in Python geschriebene event-getriebene Netzwerkengine. Die meisten gängigen Protokolle wie TCP, IMAP, SSHv3 und viele mehr werden unterstützt. Somit bietet Twisted die ideale Möglichkeit einen eigenen simplen Server zu implementieren. \\\\
\textbf{Event-Getrieben (event-based):} Die Serveranwendung befindet sich in einer Schleife und wartet auf ein Event. Dieses Event ist in diesem Fall der Connect eines Clients zum Server. Für jeden Connect wird eine neue Instanz angelegt, in welcher empfangene Daten bearbeitet werden können. Die Daten werden als String ausgewertet.

\subsection{Beispielimplementierung Webserver}
Im Folgenden wird die grundlegende Implementierung eines Webservers mit Twisted gezeigt. Für die einzelnen Funktionen des HTTP-Protokolls werden Methoden deklariert. In diesem Fall wird noch ein SSL-Kontext erzeugt, welcher die Zertifikate einliest und validiert und dafür sorgt, dass die Übertragung verschlüsselt wird.

\begin{lstlisting}[caption =Testcode Echoserver mit Twisted Framework, language=python, frame=single, breaklines=true,columns=fullflexible, commentstyle=\color{gray}\upshape, captionpos=b, numbers = left]
from twisted.web.server import Site
from twisted.web.resource import Resource
from twisted.internet import reactor
import cgi
from twisted.internet.protocol import Factory, Protocol
from twisted.internet import reactor

class Webserver(Resource):
  def render_POST(self, request):
	print cgi.escape(request.args["data"][0]))

  def render_GET(self, request):
	print cgi.escape(request.args["data"][0]))

root = Resource()
root.putChild("serv", Webserver())
factory = Site(root)
 sslContext = ssl.DefaultOpenSSLContextFactory(
     './certs/server.key', './certs/server.crt'
)
reactor.listenSSL(8000, factory, contextFactory = sslContext)
\end{lstlisting}




\subsection{Implementierung Webserver}
Der Server wird in einem neuen Thread gestartet, damit er beim Empfang von Daten keine anderen Abläufe aufhält. Zusätzich wird ihm eine Instanz der Klasse "RecvdData" übergeben. Diese verarbeitet die empfangene Nachricht. 
Anhand der ":" werden die empfangenen Daten gesplittet und in ein Array abgelegt. Zur besseren Verständlichkeit werden die Werte in einzelne Variablen gespeichert. \\
Im Anschluss wird das Übertragene Passwort und die Korrektheit der Daten überprüft. Falls beides Korrekt ist, so werden die Daten anhand ihres "Control"-Feldes ausgewertet. \\
Bevor tatsächlich LEDs angesteuert werden, wird überprüft ob die Farbwert im gültigen Bereich (0-255) liegen und ob die Angabe der LED-Nummer korrekt ist.\\

\begin{lstlisting}[caption =Implementierung des Webservers in Python, language=python, frame=single, breaklines=true,columns=fullflexible, commentstyle=\color{gray}\upshape, captionpos=b, numbers = left]
#!/usr/bin/python
# -*- coding: utf-8 -*-
################################################
# Author: Timo Höting       				   #
# Mail: mail[at]timohoeting.de  			   #
################################################

from twisted.web.server import Site
from twisted.web.resource import Resource
from twisted.internet import reactor
import cgi
from twisted.internet.protocol import Factory, Protocol
from twisted.internet import reactor
import hashlib
from ConfigReader import *
import threading
from twisted.internet import reactor, ssl

class LightServer(Resource):
  def render_POST(self, request):
    message = datamanager.dataReceived(cgi.escape(request.args["data"][0]))
    if (message != ""):
        return message

class StartLightServer(threading.Thread):
  def __init__(self, d):
     threading.Thread.__init__(self)
     global datamanager
     datamanager = d

  def run(self):
     root = Resource()
     root.putChild("serv", LightServer())
     factory = Site(root)
     sslContext = ssl.DefaultOpenSSLContextFactory(
         './certs/server.key', './certs/server.crt'
     )
     reactor.listenSSL(8000, factory, contextFactory = sslContext)
     #reactor.listenTCP(8000, factory)
     reactor.run(installSignalHandlers=False)
\end{lstlisting}

\begin{lstlisting}[caption =Implementierung des Nachrichten-Verarbeitung in Python, language=python, frame=single, breaklines=true,columns=fullflexible, commentstyle=\color{gray}\upshape, captionpos=b, numbers = left]
#!/usr/bin/python
# -*- coding: utf-8 -*-
################################################
# Author: Timo Höting       				   #
# Mail: mail[at]timohoeting.de  			   #
################################################

import hashlib
from ConfigReader import *
import threading

class RecvdData(threading.Thread):
    def __init__(self, c):
        threading.Thread.__init__(self)
        global center
        center = c

    def dataReceived(self, data):
        # Protokoll: auth:pw:control:ledNo:rangeStart:rangeEnd:red:green:blue:modus:effectcode:config:hashv
        # Beispiel: admin:w:X00:1:0:0:10:10:10:0:0:w-w:58acb7acccce58ffa8b953b12b5a7702bd42dae441c1ad85057fa70b
        # Ermoeglicht Zuweisung von Farben und Effekten
        # Ermöglicht Abruf von aktuellem Status des Systems und der LEDs
        #
        # Ankommende String bei ":" aufsplitten und in Array a[] Speichern:
        a = data.split(':')
        print a
        if len(a) > 1:
            auth = 		a[0]
            pw = 		a[1]
            control = 	a[2]
            ledNo = 	a[3]
            rangeStart = a[4]
            rangeEnd = 	a[5]
            red = 		a[6]
            green = 	a[7]
            blue = 		a[8]
            modus = 	a[9]
            effectcode = a[10]
            config = a[11]
            hashv = a[12]
            data = auth + pw + control + ledNo + rangeStart + rangeEnd + red + green + blue + modus + effectcode + config
            data = data.rstrip('\n')
            data = data.rstrip('\r')
            if (self.checkAuthentification(auth, pw) & self.checkTransmissionData(data, hashv)):
                if control == 'X00':
                    ## Alle LEDs ausschalten
                    center.clearPixel()
                elif control == 'X01':
                    ## Eine LED anschalten
                    self.lightUpOneLED(int(ledNo), int(red), int(green), int(blue))
                elif control == 'X02':
                    ## LED Bereich anschalten
                    self.lightUpLEDRange(int(rangeStart), int(rangeEnd), int(red), int(green), int(blue))
                elif control == 'X03':
                    ## Eine Farbe für alle LED
                    self.lightUpAllLED(int(red), int(green), int(blue))
                elif control == 'X04':
                    ## Effekt alle LEDs
                    self.effectLED(effectcode)
                elif control == 'X05':
                    ## Modus des Systems
                    self.changeModus(int(modus))
                elif control == 'X06':
                    ## Systemstatus als JSON an den Client
                    return self.sendStatus()
                elif control == 'X07':
                    ## Status der einzelnen LEDs senden
                    return self.sendLEDStatus()
                elif control == 'X08':
                    ## Konfiguration ändern
                    self.changeConfiguration(config)
                elif control == 'X09':
                    ## Login
                    return "LOGIN:TRUE"
            else:
                print center.writeLog('Übertragung fehlerhaft')

    def changeModus(self, modus):
        if modus >= 0 & modus < 4:
            center.setModus(modus)

    def lightUpOneLED(self, ledNo, red, green, blue):
        # Eine einzelne LED mit den o.g. RGB-Werten dauerhaft anschalten
        a = self.checkColorRange(red)
        b = self.checkColorRange(green)
        c = self.checkColorRange(blue)
        d = self.checkRange(ledNo)
        if ( a & b & c & d):
            center.lightUpOneLED(ledNo, red, green, blue)

    def lightUpLEDRange(self, rangeStart, rangeEnd, red, green, blue):
        # Einen Bereich von LEDs mit den o.g. RGB-Werten
        # dauerhaft einschalten
        # Bereich muss ueberprueft werden mit checkRange()
        a = self.checkColorRange(red)
        b = self.checkColorRange(green)
        c = self.checkColorRange(blue)
        d = self.checkRange(rangeStart)
        e = self.checkRange(rangeEnd)
        if ( a & b & c & d & e):
            center.rangePixel(rangeStart, rangeEnd, red, green, blue)

    def lightUpAllLED(self, red, green, blue):
        # Alle LEDs mit den o.g. RGB-Werten
        # dauerhaft einschalten
        # Bereich muss ueberprueft werden mit checkRange()
        a = self.checkColorRange(red)
        b = self.checkColorRange(green)
        c = self.checkColorRange(blue)
        if ( a & b & c):
            center.lightUpAllLED(red, green, blue)

    def effectLED(self, code):
        # Effekte auf einer LED aktivieren
        center.effectLED(code)

    def checkRange(self, ledNo):
        # Ueberprueft ob die uebergeben LED-Nummer ueberhaupt im
        # gueltigen Bereich liegt
        # Es wird der Eintrag 'number' aus dem Config-File geladen
        reader = ConfigReader()
        number = int(reader.getNumberOfLED())
        if ( ledNo >= 0 & ledNo < number):
            return True
        else:
            return False

    def checkColorRange(self, color):
        # Überprüfung ob Farbwert im gültigen Bereich liegt
        if (color >= 0 & color <= 255):
            return True
        return False

    def checkAuthentification(self, auth, pw):
        # TODO
        # Authentifizierung überprüfen
        # Eingabewert ist das Passwort aus der Übertragung
        # Dieses wird gehasht und mit dem in der Konfiguration gespeicherten
        # Hashwert verglichen
        reader = ConfigReader()
        hashv = reader.getHashPass()
        pw = hashlib.sha224(auth).hexdigest()
        if ( pw == hashv ):
            return True
        return False

    def checkTransmissionData(self, data, check):
        # Korrektheit der Übertragung mittels Hashvergleich feststellen
        # Eingabewert sind die gesamten Daten der Übertragung
        hashdata = hashlib.sha224(data).hexdigest()
        check = check.rstrip('\n')
        check = check.rstrip('\r')
        if ( hashdata == check ):
            return True
        # TODO Für Testübertragung return immer True
        return True

    def sendStatus(self):
        # Status des Systems senden
        reader = ConfigReader()
        message = 'STATUS:{"ledcount":"' + reader.getNumberOfLED() + '","motionport1":"' + reader.getMotionPin1() + '","motionport2":"' + reader.getMotionPin2() + '","ftp_url":"' + reader.getFTP() + '","camavaible":"'
        message = message + reader.camAvaible() + '","cam_url":"' + reader.camURL() + '","cam_url_short":"' + reader.camShortURL() + '","timeperiod":"' + reader.getTimePeriod() + '"}'
        print message
        return str(message)

    def sendLEDStatus(self):
        # Farbwerte aller einzelnen LEDs senden
        ledstatus = center.getLEDStatusAsJson()
        return ledstatus

    def changeConfiguration(self, config):
        b = config.split('--')
        key = b[0]
        value = b[1]
        center.changeConfiguration(key, value)
\end{lstlisting}

\subsection{Hashfunktion}
Es wird zu zweierlei Zwecken eine Hashfunktion eingesetzt. Zum einen um die Korrektheit der Übertragung zu überprüfen und zum Anderen um ein Passwort zur Authentifizierung verwenden zu können. Dieses wird als Wort übertragen, auf dem Server aber nur als Hash-Wert abgespeichert. Falls es also jemand schafft die Konfirgurationsdatei abzugreifen, so ist der Passworthash nichts wert.\\\\
\textbf{Hash-Funktion:} Eine Hashfunktion ist eine Einwegfunktion die aus einer großen Eingabemenge, eine kleinere Zielmenge generiert.  Die Ausgabe muss für die selbe Eingabe immer gleich sein. Jedoch soll bei der kleinsten Änderung der Eingabe, eine möglichst große Veränderung in der Ausgabe auftreten.

\section{Kamera}
\subsection{PI-Kamera vs. Netzwerkkamera}
Bei Auslösen des System im Überwachungsmodus soll ein aktuelles Bild der Überwachungskamera an das jeweilige Sartphone gepusht werden. Es gibt zwei mögliche Kameratechniken, entweder eine direkt an den Raspberry Pi Angeschlossene oder eine, die im Netzwerk erreichbar ist. 
\paragraph{Raspberry Pi Cam} \\
Die Kameras für den Raspberry Pi können direkt an das Gerät angeschlossen werden. Meistens werden sie direkt über die GPIO Pins verbunden. Der Vorteil dieser Kameras ist, dass sie keine externe Stromversorgung benötigen und durch viele verschiedene Frameworks leicht anpassbar und verwaltbar sind. Der große Nachteil ist allerdings, dass die Kamera an dem Raspberry Pi angeschlossen werden muss, auf welchem auch der Server läuft. Da dieser aber möglich wettergeschützt (im Außenbereich) oder unauffällig (im Innenbereich) angebracht ist, lässt sich von diesen Positionen kaum eine effektive Überwachung realisieren. \\
Als Beispiel wäre ein von der Raspberry Pi Foundation empfohlene Kamera zu nennen: //TODO Beispielkamera
\paragraph{Netzwerkkamera} \\
Eine Netzwerkkamera oder auch IP-Kamera genannt befindet sich im Netzwerk und kann über eine Website oder App eingesehen und gesteuert werden. Der Vorteil ist, dass sie sich irgendwo befinden kann, solange sie im selben Netzwerk ist. Somit kann zum Beispiel eine wetterfeste Kamera im Außenbereich angebracht werden und der Server kann sich im geschützten Innenbereich befinden. \\
Der Nachteil besteht bei IP-Kameras darin, dass es keine einheitliche API zum Abgreifen des Videomaterials gibt. Eine mögliche Lösung wäre das Laden der HTML Seite über einen HTTP-Request und darauffolgend das Ausfiltern des Bildmaterials. Über diese Variante kann aber kein Video sondern nur temporäre Bilder geladen werden. Dies würde aber für eine Notification auf dem Smartphone ausreichen. \\\\
Für dieses Projekt wird eine IP-Kamera aufgrund von oben genannten Vorteilen verwendet.
\subsection{Ansteuerung}
\paragraph{Testcode HTTP-Request}
\paragraph{Implementierung}

\section{Konfiguration und Installation}
\subsection{Konfiguration}
\subsection{Installation}

\section{iOS App}
\subsection{Konzept}
\subsection{...}
\subsection{...}

\chapter{Praktische Umsetzung}

\chapter{Kostenaufstellung}

\chapter{Fazit}

\addcontentsline{toc}{chapter}{Abbildungsverzeichnis}
\listoffigures

\begin{thebibliography}{xxxxxxxxxxxxxxxxxxx}
 \bibitem[I.]{eventbased}"'SWR Info - Zahlen, Daten, Fakten über den SWR"', \url{http://www.swr.de/unternehmen/unternehmen/kennzahlen/kennzahlen-organisation/-/id=12213420/did=12302978/nid=12213420/eqq46v/index.html}, 13.12.2013
%\bibitem[Thema]{Bezeichner}"'Überschrift"', \url{LINK}, Datum
\end{thebibliography}

 \end{document}

 