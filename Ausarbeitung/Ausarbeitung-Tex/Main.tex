\documentclass[12pt,a4paper]{scrreprt}
 \usepackage{ngerman}
 \usepackage[utf8]{inputenc}
 \usepackage[T1]{fontenc}
\usepackage{pdfpages}
\usepackage{url}
\graphicspath{{/Users/Timo/Documents/DHBW/Semester4/Projektarbeit/}}
\usepackage{graphicx}
\usepackage{wrapfig}
\usepackage{geometry} 
\geometry{a4paper, top=25mm, left=25mm, right=25mm, bottom=25mm, headsep=15mm, 
  footskip=12mm}
\usepackage{textcomp}
% Header
\usepackage{fancyhdr}
\pagestyle{fancy}
% Header für Seiten ohne Chapter
\fancyhf{}
\fancyhead[L]{Studienarbeit Timo Höting \\ DHBW Karlsruhe}
   \fancyhead[R]{\includegraphics[scale=0.3]{./data/dhbwlogo.jpg} } 
\fancyfoot[C]{\thepage}
\fancypagestyle{plain}{
% Header für Seiten mit Chapter
\fancyhf{}
   \fancyhead[L]{Studienarbeit Timo Höting \\ DHBW Karlsruhe}
   \fancyhead[R]{\includegraphics[scale=0.3]{./data/dhbwlogo.jpg} } 
\fancyfoot[C]{\thepage}
}

 \begin{document}
\begin{titlepage}
\begin{figure}
\makebox[\textwidth]{\includegraphics[page={1},width=\paperwidth]{./data/deckblatt.pdf}} \\
\end{figure}
\end{titlepage}
\clearpage
\begin{figure}
\makebox[\textwidth]{\includegraphics[page={2},width=\paperwidth]{./data/deckblatt.pdf}} \\
\end{figure}
\thispagestyle{empty} %Head löschen
 \tableofcontents
\thispagestyle{empty} 
\chapter{Einleitung}
asdf
\section{Projektbeschreibung}
asdf
\section{Teilprojekte}
asdf
\chapter{Hauptteil}
\section{LED-Pixel}
\subsection{Bewertungskriterien}
\subsection{Evaluierung}
\subsection{Teststellung}

\section{Bewegungssensor}
\subsection{Bewertungskriterien}
\subsection{Evaluierung}
\subsection{Fazit}
\subsection{Teststellung}

\section{Python-Server und Protokoll}
\subsection{Protokoll}
\subsection{Framework}
\subsection{Testcode}
\subsection{Implementierung}
\subsection{Klassen und ihre Funktionen}
\subsection{Hashfunktion}

\section{Verschlüsselung}
\subsection{SSL vs. TLS}
\subsection{Vor- und Nachteile TLS}
\subsection{TLS Handshake}
\subsection{Zertifikat und Key}
\subsection{Beispielcode Server}
\subsection{Wireshark Trace}

\section{Kamera}
\subsection{PI-Kamera vs. Netzwerkkamera}
\subsection{Ansteuerung}

\section{iOS App}
\subsection{Konzept}
\subsection{...}
\subsection{...}

\chapter{Praktische Umsetzung}

\chapter{Kostenaufstellung}

\chapter{Fazit}


 \end{document}

 